% Options for packages loaded elsewhere
\PassOptionsToPackage{unicode}{hyperref}
\PassOptionsToPackage{hyphens}{url}
%
\documentclass[
]{article}
\usepackage{amsmath,amssymb}
\usepackage{lmodern}
\usepackage{ifxetex,ifluatex}
\ifnum 0\ifxetex 1\fi\ifluatex 1\fi=0 % if pdftex
  \usepackage[T1]{fontenc}
  \usepackage[utf8]{inputenc}
  \usepackage{textcomp} % provide euro and other symbols
\else % if luatex or xetex
  \usepackage{unicode-math}
  \defaultfontfeatures{Scale=MatchLowercase}
  \defaultfontfeatures[\rmfamily]{Ligatures=TeX,Scale=1}
\fi
% Use upquote if available, for straight quotes in verbatim environments
\IfFileExists{upquote.sty}{\usepackage{upquote}}{}
\IfFileExists{microtype.sty}{% use microtype if available
  \usepackage[]{microtype}
  \UseMicrotypeSet[protrusion]{basicmath} % disable protrusion for tt fonts
}{}
\makeatletter
\@ifundefined{KOMAClassName}{% if non-KOMA class
  \IfFileExists{parskip.sty}{%
    \usepackage{parskip}
  }{% else
    \setlength{\parindent}{0pt}
    \setlength{\parskip}{6pt plus 2pt minus 1pt}}
}{% if KOMA class
  \KOMAoptions{parskip=half}}
\makeatother
\usepackage{xcolor}
\IfFileExists{xurl.sty}{\usepackage{xurl}}{} % add URL line breaks if available
\IfFileExists{bookmark.sty}{\usepackage{bookmark}}{\usepackage{hyperref}}
\hypersetup{
  pdftitle={Tugas Modul 7},
  pdfauthor={Rivanka Desya},
  hidelinks,
  pdfcreator={LaTeX via pandoc}}
\urlstyle{same} % disable monospaced font for URLs
\usepackage[margin=1in]{geometry}
\usepackage{color}
\usepackage{fancyvrb}
\newcommand{\VerbBar}{|}
\newcommand{\VERB}{\Verb[commandchars=\\\{\}]}
\DefineVerbatimEnvironment{Highlighting}{Verbatim}{commandchars=\\\{\}}
% Add ',fontsize=\small' for more characters per line
\usepackage{framed}
\definecolor{shadecolor}{RGB}{248,248,248}
\newenvironment{Shaded}{\begin{snugshade}}{\end{snugshade}}
\newcommand{\AlertTok}[1]{\textcolor[rgb]{0.94,0.16,0.16}{#1}}
\newcommand{\AnnotationTok}[1]{\textcolor[rgb]{0.56,0.35,0.01}{\textbf{\textit{#1}}}}
\newcommand{\AttributeTok}[1]{\textcolor[rgb]{0.77,0.63,0.00}{#1}}
\newcommand{\BaseNTok}[1]{\textcolor[rgb]{0.00,0.00,0.81}{#1}}
\newcommand{\BuiltInTok}[1]{#1}
\newcommand{\CharTok}[1]{\textcolor[rgb]{0.31,0.60,0.02}{#1}}
\newcommand{\CommentTok}[1]{\textcolor[rgb]{0.56,0.35,0.01}{\textit{#1}}}
\newcommand{\CommentVarTok}[1]{\textcolor[rgb]{0.56,0.35,0.01}{\textbf{\textit{#1}}}}
\newcommand{\ConstantTok}[1]{\textcolor[rgb]{0.00,0.00,0.00}{#1}}
\newcommand{\ControlFlowTok}[1]{\textcolor[rgb]{0.13,0.29,0.53}{\textbf{#1}}}
\newcommand{\DataTypeTok}[1]{\textcolor[rgb]{0.13,0.29,0.53}{#1}}
\newcommand{\DecValTok}[1]{\textcolor[rgb]{0.00,0.00,0.81}{#1}}
\newcommand{\DocumentationTok}[1]{\textcolor[rgb]{0.56,0.35,0.01}{\textbf{\textit{#1}}}}
\newcommand{\ErrorTok}[1]{\textcolor[rgb]{0.64,0.00,0.00}{\textbf{#1}}}
\newcommand{\ExtensionTok}[1]{#1}
\newcommand{\FloatTok}[1]{\textcolor[rgb]{0.00,0.00,0.81}{#1}}
\newcommand{\FunctionTok}[1]{\textcolor[rgb]{0.00,0.00,0.00}{#1}}
\newcommand{\ImportTok}[1]{#1}
\newcommand{\InformationTok}[1]{\textcolor[rgb]{0.56,0.35,0.01}{\textbf{\textit{#1}}}}
\newcommand{\KeywordTok}[1]{\textcolor[rgb]{0.13,0.29,0.53}{\textbf{#1}}}
\newcommand{\NormalTok}[1]{#1}
\newcommand{\OperatorTok}[1]{\textcolor[rgb]{0.81,0.36,0.00}{\textbf{#1}}}
\newcommand{\OtherTok}[1]{\textcolor[rgb]{0.56,0.35,0.01}{#1}}
\newcommand{\PreprocessorTok}[1]{\textcolor[rgb]{0.56,0.35,0.01}{\textit{#1}}}
\newcommand{\RegionMarkerTok}[1]{#1}
\newcommand{\SpecialCharTok}[1]{\textcolor[rgb]{0.00,0.00,0.00}{#1}}
\newcommand{\SpecialStringTok}[1]{\textcolor[rgb]{0.31,0.60,0.02}{#1}}
\newcommand{\StringTok}[1]{\textcolor[rgb]{0.31,0.60,0.02}{#1}}
\newcommand{\VariableTok}[1]{\textcolor[rgb]{0.00,0.00,0.00}{#1}}
\newcommand{\VerbatimStringTok}[1]{\textcolor[rgb]{0.31,0.60,0.02}{#1}}
\newcommand{\WarningTok}[1]{\textcolor[rgb]{0.56,0.35,0.01}{\textbf{\textit{#1}}}}
\usepackage{graphicx}
\makeatletter
\def\maxwidth{\ifdim\Gin@nat@width>\linewidth\linewidth\else\Gin@nat@width\fi}
\def\maxheight{\ifdim\Gin@nat@height>\textheight\textheight\else\Gin@nat@height\fi}
\makeatother
% Scale images if necessary, so that they will not overflow the page
% margins by default, and it is still possible to overwrite the defaults
% using explicit options in \includegraphics[width, height, ...]{}
\setkeys{Gin}{width=\maxwidth,height=\maxheight,keepaspectratio}
% Set default figure placement to htbp
\makeatletter
\def\fps@figure{htbp}
\makeatother
\setlength{\emergencystretch}{3em} % prevent overfull lines
\providecommand{\tightlist}{%
  \setlength{\itemsep}{0pt}\setlength{\parskip}{0pt}}
\setcounter{secnumdepth}{-\maxdimen} % remove section numbering
\ifluatex
  \usepackage{selnolig}  % disable illegal ligatures
\fi

\title{Tugas Modul 7}
\author{Rivanka Desya}
\date{11/8/2021}

\begin{document}
\maketitle

\hypertarget{r-markdown}{%
\subsection{R Markdown}\label{r-markdown}}

\begin{enumerate}
\def\labelenumi{\arabic{enumi}.}
\tightlist
\item
  Gunakan as\_tibble untuk mengkonversi tabel dataset ``US murders''
  dalam bentuk tibble dan simpan dalam objek baru bernama
  `murders\_tibble'.
\end{enumerate}

\begin{Shaded}
\begin{Highlighting}[]
\FunctionTok{library}\NormalTok{(tidyverse)}
\end{Highlighting}
\end{Shaded}

\begin{verbatim}
## -- Attaching packages --------------------------------------- tidyverse 1.3.1 --
\end{verbatim}

\begin{verbatim}
## v ggplot2 3.3.5     v purrr   0.3.4
## v tibble  3.1.4     v dplyr   1.0.7
## v tidyr   1.1.3     v stringr 1.4.0
## v readr   2.0.1     v forcats 0.5.1
\end{verbatim}

\begin{verbatim}
## -- Conflicts ------------------------------------------ tidyverse_conflicts() --
## x dplyr::filter() masks stats::filter()
## x dplyr::lag()    masks stats::lag()
\end{verbatim}

\begin{Shaded}
\begin{Highlighting}[]
\FunctionTok{library}\NormalTok{(dslabs)}
\FunctionTok{data}\NormalTok{(murders)}
\FunctionTok{as\_tibble}\NormalTok{(murders) }\SpecialCharTok{\%\textgreater{}\%} \FunctionTok{class}\NormalTok{()}
\end{Highlighting}
\end{Shaded}

\begin{verbatim}
## [1] "tbl_df"     "tbl"        "data.frame"
\end{verbatim}

\begin{Shaded}
\begin{Highlighting}[]
\NormalTok{murders\_tibble }\OtherTok{\textless{}{-}} \FunctionTok{as\_tibble}\NormalTok{(murders) }\SpecialCharTok{\%\textgreater{}\%} \FunctionTok{class}\NormalTok{() }
\end{Highlighting}
\end{Shaded}

\begin{enumerate}
\def\labelenumi{\arabic{enumi}.}
\setcounter{enumi}{1}
\tightlist
\item
  Gunakan fungsi group\_by untuk mengkonversi dataset ``US murders''
  menjadi sebuah tibble yang dikelompokkan berdasarkan `region'.
\end{enumerate}

\begin{Shaded}
\begin{Highlighting}[]
\FunctionTok{as\_tibble}\NormalTok{(murders) }\SpecialCharTok{\%\textgreater{}\%} \FunctionTok{group\_by}\NormalTok{(region)}
\end{Highlighting}
\end{Shaded}

\begin{verbatim}
## # A tibble: 51 x 5
## # Groups:   region [4]
##    state                abb   region    population total
##    <chr>                <chr> <fct>          <dbl> <dbl>
##  1 Alabama              AL    South        4779736   135
##  2 Alaska               AK    West          710231    19
##  3 Arizona              AZ    West         6392017   232
##  4 Arkansas             AR    South        2915918    93
##  5 California           CA    West        37253956  1257
##  6 Colorado             CO    West         5029196    65
##  7 Connecticut          CT    Northeast    3574097    97
##  8 Delaware             DE    South         897934    38
##  9 District of Columbia DC    South         601723    99
## 10 Florida              FL    South       19687653   669
## # ... with 41 more rows
\end{verbatim}

\begin{enumerate}
\def\labelenumi{\arabic{enumi}.}
\setcounter{enumi}{2}
\tightlist
\item
  Tulis script tidyverse yang menghasilkan output yang sama dengan
  perintah berikut:
\end{enumerate}

exp(mean(log(murders\$population)))

Gunakan operator pipe sehingga setiap fungsi dapat dipanggil tanpa
menambahkanargumen. Gunakan dot operator untuk mengakses populasi.

\begin{Shaded}
\begin{Highlighting}[]
\FunctionTok{library}\NormalTok{(dplyr)}
\NormalTok{murders }\SpecialCharTok{\%\textgreater{}\%} \FunctionTok{pull}\NormalTok{(population) }\SpecialCharTok{\%\textgreater{}\%}\NormalTok{ log }\SpecialCharTok{\%\textgreater{}\%}\NormalTok{ mean }\SpecialCharTok{\%\textgreater{}\%}\NormalTok{ exp}
\end{Highlighting}
\end{Shaded}

\begin{verbatim}
## [1] 3675209
\end{verbatim}

\begin{enumerate}
\def\labelenumi{\arabic{enumi}.}
\setcounter{enumi}{3}
\tightlist
\item
  Gunakan map\_df untuk membuat data frame yang terdiri dari tiga kolom:
  `n', `s\_n', dan `s\_n\_2'. Kolom pertama harus berisi angka 1 hingga
  100. Kolom kedua dan ketiga masingmasing harus berisi penjumlahan 1
  hingga n, dimana n menyatakan jumlah baris.
\end{enumerate}

\begin{Shaded}
\begin{Highlighting}[]
\FunctionTok{library}\NormalTok{(purrr)}
\NormalTok{compute\_s\_n }\OtherTok{\textless{}{-}} \ControlFlowTok{function}\NormalTok{(n)\{ }
\NormalTok{ x }\OtherTok{\textless{}{-}} \DecValTok{1}\SpecialCharTok{:}\NormalTok{n }
 \FunctionTok{sum}\NormalTok{(x) }
\NormalTok{\} }
\NormalTok{n }\OtherTok{\textless{}{-}} \DecValTok{1}\SpecialCharTok{:}\DecValTok{100} 
\NormalTok{s\_n }\OtherTok{\textless{}{-}} \FunctionTok{sapply}\NormalTok{(n, compute\_s\_n)}
\NormalTok{compute\_s\_n }\OtherTok{\textless{}{-}} \ControlFlowTok{function}\NormalTok{(n)\{ }
\NormalTok{ x }\OtherTok{\textless{}{-}} \DecValTok{1}\SpecialCharTok{:}\NormalTok{n }
 \FunctionTok{tibble}\NormalTok{(}\AttributeTok{sum =} \FunctionTok{sum}\NormalTok{(x)) }
\NormalTok{\} }
\NormalTok{s\_n }\OtherTok{\textless{}{-}} \FunctionTok{map\_df}\NormalTok{(n, compute\_s\_n)}
\FunctionTok{as\_tibble}\NormalTok{(s\_n)}
\end{Highlighting}
\end{Shaded}

\begin{verbatim}
## # A tibble: 100 x 1
##      sum
##    <int>
##  1     1
##  2     3
##  3     6
##  4    10
##  5    15
##  6    21
##  7    28
##  8    36
##  9    45
## 10    55
## # ... with 90 more rows
\end{verbatim}

\end{document}
